\section{Big Data}
Big data often refers to exceedingly large and complex data, often characterized by three V's: volume, velocity, and variety. Depending on the organisation, when talking about big data, the volume can be anywhere from terabytes to petabytes. Companies working with big data usually also get the data at a high velocity, and depending on the company that uses the data, the company might also need it to move at a high velocity. Data can come from several different types of sources. Due to this, it may need to be processed in near real-time to be used effectively. When gathering data, the data can come in a variety of formats and a lot of these data types are unstructured or semi-unstructured \cite{big_data_oracle}.

\subsection{Problems with big data}
Some of the problems with big data is its massive volume which makes it hard to store. Furthermore, data has to be processed, from raw information to useful information, in order to be efficiently used. This can be a problem as it takes a lot of work. A data scientist can spend between 50 to 80 percent of their time on just preparing data \cite{big_data_oracle}. This takes a lot of processing power that can be costly to have. Another problem when it comes to big data is security, which is a significant concern for organisations. When information is left unencrypted it is at a high risk of being stolen or compromised by a malicious actor. This can become an issue if not handled correctly, as one still needs to access the data without putting it at risk.

\subsection{Why is grid computing relevant for big data}
Grid computing can be a good solution for the problems big data has, as it enables the processing of massive volumes of data in a distributed manner. Considering the massive volume, velocity, and variety of data, processing it all could be too much for a single machine. While giving a lot of processing power, grid computing could also result in faster processing times and more efficient use of resources. This becomes especially relevant in organisations where information needs real-time analysis to keep up.