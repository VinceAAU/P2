\subsection{Authentication \& Authorisation}
Sending data from one node to another in a grid computing system can cause problems. These transfers can be the target of malicious attacks or nodes can present themselves as a trusted node, without being a trusted node itself. This makes it important to authenticate the different nodes, so that the identity of each node is checked and validated. 

\subsubsection{Certificate Authority (CA)}
One way to ensure that both nodes involved in a data transfer are authenticated is by using a certificate authority (CA). This is a trusted node that is responsible for checking and entrusting new nodes. This is done with a two-key encryption system with a public and a private key. Every node has its own set of keys. The private key is kept safe by that node, and no one else should know that key, while the public key is known by anyone who wants it. This is also the case for the CA node. When adding a new node to the network, that node would then provide a certificate with the information about the node, containing the public key of the node, to the CA node. The CA node would then check the information and if everything is in order, it would then sign it with its private key. This means that from now on the new node would be able to show this certificate and other nodes could verify that this node has been checked by the CA node since they can use the CA public key to decrypt the encrypted signature of the CA node. The node that checked the signature can then use the public key on the certificate to encrypt and transfer data to the node that had the certificate. This certification can be shared and copied without any risk, as only the node that originally got the certificate can use it. This is because anything encrypted by the public key on the certificate can only be decrypted by the private key that only that one node knows \cite{IBM}.

This means that any transfer between two nodes would be as follows. Node A sends a certificate to node B. Node B checks the certificate. Node B then sends its certificate, which is verified by node A. Both nodes have now been verified. Node A will then encrypt its data, first with its own private key, then with the recipient's public key. This method ensures that only the intended recipient, in this case, node B, can access the data, as only they possess the private key that matches the public key on their certificate. It also ensures that node A is the sender, as only their public key will decrypt what was encrypted with their private key. It is therefore very hard to access the data for anyone who might have intercepted the transfer \cite{IBM}.

When dealing with larger systems, some of the tasks that the CA node handles might be handed off to another node. This node is known as a registrant authority (RA). The RA node takes care of checking whether the information on a not yet signed certificate is correct, before it is handed over to the CA node to lessen the workload of the CA node \cite{IBM}.

\subsubsection{Attribute Based Authorisation}
Different nodes in a grid may have different tasks or may be trusted on different levels. You may need to have nodes that are only workers, nodes that can provide work tasks, and nodes that can distribute tasks. This is done by granting different nodes authority to complete certain tasks. The simplest version of this is keeping track of exactly what every single node is allowed to do, so they can be granted access to these tasks after they have been authenticated. This works well for small grids but is not very scalable, as grids can have hundreds of nodes, and you then have to keep track of permissions individually. Another solution for this is to use an attribute-based authorisation system. This system is built on creating and assigning different roles. This might be a worker, a task scheduler, and a work provider. Nodes in the same group would have the same authorisation level and can therefore just be assigned as a worker. The system would then know what a worker is allowed to do without having to check or store the specific nodes' authorisation. This also has the benefit of being able to edit authorisation in one central place instead of having to change it for every single node \cite{4144613}.
