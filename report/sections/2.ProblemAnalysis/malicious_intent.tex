\subsection{Malicious Intent}
Another thing that is not necessarily inevitable, but a thing that one should worry about, is the presence of malicious intent in one's grid computing system. It is important that the system can detect and handle such intent.

In the context of this report, malicious intent is when some outside force or a node within the system, tries to disrupt the flow of the working grid. Some examples of malicious intent could be when attackers try to take the resources utilized by the grid and use them for their own gain, or when the attackers try to make a worker node create a false or invalid result, or in some way spread viruses among the nodes in the grid \cite{detecting_malicious_manipulation}.

There exist many different ways of detecting and catching malicious nodes, one method is a system-level diagnosis, where the system sends ping-messages to the nodes containing tests and through these tests, the system can detect malicious or faulty nodes. A classic example of such a diagnosis is the Preparata-Metze-Chien (PMC) diagnosis, where a test is sent out to multiple nodes and the result is then checked at the master node. If all the nodes give the same result, they are considered non-malicious, but if one gives a different result from others it is considered malicious and should be dealt with \cite{detecting_malicious_manipulation}.

Malicious intent can be seen as a form of security hacking and the motives for creating malicious nodes and hacking are very similar. One motive could be monetary gain. Attackers could achieve this in a number of ways, such as using the data that they gather to blackmail or extort money from either a corporation or a person. Another way that attackers could obtain money could be through selling the data that was obtained illegally. If they get data from a company they could sell the data to other competing companies or if they get data from a private person they could sell it to companies who would want such data \cite{why_do_hackers_hack}.