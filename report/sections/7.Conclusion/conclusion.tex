\chapter{Conclusion}\label{ch:conclusion}

In this project, the intricacies of grid computing were explored, focusing on its potential to address challenges related to big data. The aim was to gather insights into different ways of utilising and implementing grids while analysing the underlying issues that arise in the context of big data processing. This analysis led to the following problem statement:

\begin{center}
\emph{How can a web-based software solution be developed for time-efficient sorting of big data on a grid computing system?}
\end{center}
% 
A custom grid computing system, optimised for efficient sorting of tasks, was designed, developed, and tested in a browser-based environment. The performance and effectiveness of the system underwent quality testing, including unit and end-to-end testing, to verify the functionality of its components and its error-handling capabilities. On a small set of data, the grid was not capable of sorting data at a faster rate than a high-end consumer-grade computer, but whether this was due to the limitations of the master node or an inefficient implementation remains uncertain. It is, however, important to note that the sorting capabilities were assessed using relatively small data files within the context of big data. Consequently, it is also uncertain whether the program would have the same efficiency issues on larger sets of data with more tasks and workers, but regardless, the high memory requirements of the master node would make it unsustainable in practical use. However, some of the problems that must be addressed in order to enhance efficiency, have seemingly straightforward solutions. Whether these solutions would speed up the system enough for it to be considered efficient is difficult to determine without implementing them. 