\chapter{Quality Assurance}
Ensuring the quality of code units is an integral part of software development. Through writing high quality code, it is possible to guarantee certain behaviours. 

To facilitate this desired, high quality code, several measures are taken. These are mainly Code Conventions, Version Control, and Testing. These measures are described in this chapter in their respective sections.

Testing is specifically documented as this is a verifiable way of achieving the desired quality of code. This is the case, since tested code provides more certain behaviour because the behaviour is already well-defined through extensive testing. This enables the code to be used across multiple contexts without significant worry. With a solid grasp of the functionality provided by a unit of code, adding it into a larger context should be more likely to work as intended. As opposed to tested code, untested code has a high likelihood of possessing undiscovered edge cases that can cause unexpected behaviours that can interfere with the features of the program. This means testing can provide assurances in terms of quality, making a product of high quality achievable, if testing has been done in a suitable manner. 

This project focuses on two different types of testing, in order to attain confidence in regards to quality: Unit Tests, code written to test individual units of the product's code; and End-to-End Testing, features investigated in the form of manual review. The documentation of these tests will be featured in this chapter.


