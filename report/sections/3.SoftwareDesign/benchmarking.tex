\section{Considerations for Task Distribution} \label{sec:benchmarking}

Benchmarking worker nodes to find their computational power can be beneficial for reducing the communication overhead of the grid. However, this is not always the case. Knowing the capabilities of each node allows the task scheduler to assign a node an appropriately sized task. However, finding out what a worker is capable of, without spending more time than would be gained, proves a difficult challenge \cite{Charpentier_2017}.

In the case of this product, the capabilities of the nodes will be somewhat unpredictable, and the processing power of a node may vary greatly while processing an assigned task, due to other processes being opened or closed. For this reason, benchmarking the worker nodes at the start of, or during, a task was considered to optimise the task scheduler. However, the idea of benchmarking nodes continuously was discarded as it simply takes too much time to perform a benchmark, and many benchmarks would need to be done to account for processes being opened and closed \cite{Charpentier_2017}.
Benchmarking a node at the start of a task is more realistic, but it is unclear how useful the information gathered would be. The information would be used to give a node a smaller or larger task relative to other nodes in accordance with its processing power. The benefit of doing so is to reduce the amount of time spent waiting for slower nodes to finish in the grid. However, since the computational power can vary throughout the task, and a node may end up being slower than initially predicted, this can lead to the same problem regardless. \cite{Mengistu2019}

An alternative to benchmarking nodes is to simply send tasks of equal sizes to all worker nodes, regardless of their abilities. If the task is small enough, the time spent waiting for slow nodes will be smaller, which somewhat helps with the issue. However, having smaller packages will result in a larger communication overhead, as nodes will need to request new tasks more often. 