\chapter{Introduction}\label{ch:introduction}

In contemporary society information is valuable. The spread of the personal computer has created new avenues for information to emerge within people's lives. Within computer science, the word "data" refers to formalised information that can be processed by a machine. Interests, online shopping carts, social circles, and many more, are all pieces of data that can be used as a means to any number of arbitrary ends. However, using data to advance knowledge of a specific area of interest requires the data to be manipulated, for instance, sorted in specific ways to ensure a useful data representation and application \cite{data_ref}.

Given all the possible applications of data collected from people's devices, the sheer volume of data can create issues for traditional data processing techniques \cite{big_data_oracle}. Herein lies the need for more computational power to handle this processing. When expanding computational resources, it is worth considering how to handle it. It is both possible to do vertical and horizontal scaling. Vertical scaling refers to expanding the capabilities of one's computer or node, whereas horizontal scaling refers to adding new nodes to one's system to distribute the workload \cite{scaling_geeks}.


Vertical scaling, while easy in terms of installation, presents several downsides, such as the cost of acquiring new equipment and possessing a single point of failure. Horizontal scaling has the benefit of being able to expand the available processing powers with devices already in one's possession, for instance by making a grid of computational devices which all share resources. However, using a system of distributed computing, like grid computing, requires the development of resource management systems to function properly \cite{TaskSchedulingReview}.

To attain a deeper understanding of this problem and the challenges it presents, this project will explore grid computing as a form of horizontal scaling intended to achieve increased computational power, specifically, exploring how to satisfactorily answer the question: "How can we develop a grid computing system that can sort big data and verify the correctness of the result?".



